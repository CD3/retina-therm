% arara: pdflatex
% arara: pdflatex

% start with vim --server latex %
\documentclass[]{article}

\usepackage{siunitx}
\usepackage{physics}
\usepackage{graphicx}
\usepackage{fullpage}

\author{C.D. Clark III}
\title{A Green's Function based model for retinal temperature rise from laser exposure}

\begin{document}
\maketitle

The heat equation describing heat conduction in a material with source term is
\begin{equation*}
    \rho c \pdv{T}{t} = \div{k\grad T} + A(\vec{r},t).
\end{equation*}
Here, $T\qty(\vec{r},t)$ is the temperature field and $A\qty(\vec{r},t)$ is the
source term. The thermal properties, $\rho$, $c$, and $k$ (density, specific
heat, and thermal conductivity) are in general function of space and time, but
for a homogeneous media, this can be rewritten as:
\begin{equation*}
    \pdv{T}{t} = \alpha\laplacian{T} + \frac{A(\vec{r},t)}{\rho c}
\end{equation*}
where $\alpha = \frac{\kappa}{\rho c }$ is the thermal diffusivity.
The Green's function for the heat equation, expressed in Cartesian coordinates, is
\begin{equation*}
    G(x,y,z,t,x',y',z',t') = \qty(\frac{1}{4\pi \alpha (t-t')})^{3/2} e^{-\frac{(x-x')^2+(y-y')^2+(z-z')^2}{4\alpha(t-t')}}.
\end{equation*}
This function describes the thermal response of the media to an instantaneous
point source delivered at the point $(x',y',z')$ and time $t'$, and it can
be used to compute the response of the media to the source term $A$,
\begin{equation*}
  T(x,y,z,t) = \int_0^t \int_{-\infty}^{\infty} \int_{-\infty}^{\infty} \int_{-\infty}^{\infty} G(x,y,z,t;x',y',z',t') \frac{A(x',y',z',t')}{\rho c} \dd x' \dd y' \dd z' \dd t'
\end{equation*}
A laser beam incident on a thin, linear absorbing layer of tissue will produce a spatial source term
\begin{equation*}
    A(x',y',z') = \mu_a E(x',y') e^{-\mu_a z'} = \mu_a E_0 \bar{E(x',y') e^{-\mu_a z'}}
\end{equation*}
while the laser is on,
where $\bar{E}$ denotes the normalized beam profile ($\bar{E}(0,0,) = 0$) and $E_0$ is the irradiance at the center of the beam.
For a circular flat top beam of radius $R$, this is
\begin{equation*}
    \bar{E}(x',y') = \begin{cases}
        1 & x'^2 + y'^2 \le R^2 \\
        0 & x'^2 + y'^2 > R^2
    \end{cases}
\end{equation*}
For a circular Gaussian beam with 1/e radius $\sigma$, this
\begin{equation*}
    \bar{E}(x',y') = e^{-\frac{x'^2 + y'^2}{\sigma^2}}
\end{equation*}
Let the absorbing layer have a thickness $d$ and span from $z_0$ to $z_0 + d$. Then the temperature
by the absorbing layer while the laser is on will be
\begin{equation*}
  T(x,y,z,t) = \int_0^t \int_{z_0}^{z_0 + d} \int_{-\infty}^{\infty} \int_{-\infty}^{\infty} \qty(\frac{1}{4\pi \alpha t'})^{3/2} e^{-\frac{(x-x')^2+(y-y')^2+(z-z')^2}{4\alpha t'}} \frac{\mu_a E(x',y') e^{-\mu_a z'}}{\rho c} \dd x' \dd y' \dd z' \dd t',
\end{equation*}
where we have used the fact that $A$ does not depend on time to simplify the time-dependence.
The axial integral can be carried out analytically,
\begin{equation*}
    \int_{z_0}^{z_0 + d} e^{-\frac{(z-z')^2}{4\alpha t'}} e^{-\mu_a z'} \dd z'
\end{equation*}
Let $\beta = \frac{1}{\sqrt{4 \alpha t'}}$. Then,
\begin{equation*}
    \int_{z_0}^{z_0 + d} e^{-\beta^2 (z-z')^2-\mu_a z'} \dd z' =
    \int_{z_0}^{z_0 + d} e^{-\qty(\beta^2 z^2 + \beta^2z'^2 -2\beta^2zz'+\mu_a z')} \dd z'
\end{equation*}
Completing the square for the exponent
\begin{equation*}
  \beta^2 z^2 + \beta^2 z'^2 -2\beta^2 z z' + \mu_a z' = \qty(\beta z' + \frac{\mu_a}{2\beta})^2 - \qty(\frac{\mu_a}{2\beta})^2 = \qty(\beta z' + \frac{\mu_a}{2\beta} - \beta z)^2 - \qty(\frac{\mu_a}{2\beta})^2  + \mu_a z
\end{equation*}
gives
\begin{align*}
  e^{-\mu_a z} e^{\mu_a^2/2\beta^2}\int_{z_0}^{z_0 + d} e^{-\qty(\beta z' + \frac{\mu_a}{2\beta} - \beta z)^2} \dd z'
= e^{-\mu_a z} e^{\mu_a^2/2\beta^2}\int_{\beta z_0 + \mu_a/2\beta - \beta z}^{\beta(z_0 + d) + \mu_a/2\beta - \beta z} e^{-u^2} \frac{\dd u'}{\beta}
\end{align*}
with $u = \beta z' + \mu_a/2\beta - \beta z$, which can be integrated using definition of the error function
\begin{equation*}
  \erf(x) = \frac{2}{\sqrt{\pi}} \int_0^x e^{-u^2} \dd u,
\end{equation*}
to give
\begin{align*}
  \frac{\sqrt{\pi}}{2\beta} e^{-\mu_a z} e^{\mu_a^2/2\beta^2}\qty[\erf\qty(\beta(z_0 + d - z) + \frac{\mu_a}{2\beta} ) - \erf\qty(\beta(z_0 - z) + \frac{\mu_a}{2\beta}) ] \\
  = \sqrt{\pi\alpha t'} e^{-\mu_a z} e^{\alpha t'\mu_a^2}\qty[\erf\qty(\frac{z_0 + d - z}{\sqrt{4\alpha t'}} + \sqrt{\alpha t'}\mu_a ) - \erf\qty(\frac{z_0 - z}{\sqrt{4\alpha t'}} + \sqrt{\alpha t'}\mu_a ) ]
\end{align*}
To simplify transverse integrals we consider only the temperature along the $z$ axis. Then
we need to evaluate
\begin{equation*}
    \int \int \bar{E}(x',y') e^{-\frac{x'^2 + y'^2}{4\alpha t'}} \dd x' \dd y'
\end{equation*}
For circularly symmetric beams, we can switch to polar coordinates and
integrate the azimuthal angle,
\begin{equation*}
    \int_0^{2\pi} \int_0^\infty \bar{E}(r') e^{-\frac{r'^2}{4\alpha t'}} r'\dd r' \dd \theta'
    = 2\pi \int_0^\infty  \bar{E}(r') e^{-\frac{r'^2}{4\alpha t'}} r'\dd r'
\end{equation*}
To evaluate radial integral, we need to specify the beam profile. For a flat top
beam of radius $R$, we will have
\begin{equation*}
  2\pi \int_0^R e^{-\frac{r'^2}{4\alpha t'}} r'\dd r'
  =  4\pi \alpha t' \qty[1 - e^{-R^2/4\alpha t'}]
\end{equation*}
For a Gaussian beam with 1/e radius $\sigma$ clipped by a circular aperture of radius $R$, we will have
\begin{equation*}
2\pi \int_0^R e^{-\frac{r'^2}{\sigma^2}} e^{-\frac{r'^2}{4\alpha t'}} r'\dd r'
=  \frac{\pi}{1/4\alpha t' + 1/\sigma^2} \qty[1 - e^{-\frac{R^2}{4\alpha t' + \sigma^2}}]
=  \frac{4 \pi \alpha t' \sigma^2}{4\alpha t' + \sigma^2} \qty[1 - e^{-\frac{R^2}{4\alpha t' + \sigma^2}}]
.
\end{equation*}
For an unclipped beam, $R = \infty$.

Putting this all together, for a flat top beam we have
\begin{align}
  \label{eq:flattop_temp}
  T(r=0, z,t) =& \frac{\mu_aE_0}{\rho c} \int_0^t \qty(\frac{1}{4\pi \alpha t'})^{3/2} \qty{\int_{z_0}^{z_0 + d}e^{-\frac{(z-z')^2}{4\alpha t'}} e^{-\mu_a z'} \dd z'} \qty{2\pi\int_{0}^{R} e^{-\frac{r'^2}{4\alpha t'}}r'\dd r'}\dd t' \nonumber \\
  =& \frac{\mu_a E_0}{\rho c} \int_0^t \qty(\frac{1}{4\pi \alpha t'})^{3/2}  \nonumber \\
  \times & \qty{\sqrt{\pi\alpha t'} e^{-\mu_a z} e^{\alpha t'\mu_a^2}\qty[\erf\qty(\frac{z_0 + d - z}{\sqrt{4\alpha t'}} + \sqrt{\alpha t'}\mu_a ) - \erf\qty(\frac{z_0 - z}{\sqrt{4\alpha t'}} + \sqrt{\alpha t'}\mu_a ) ]} \nonumber \\
  \times & \qty{4\pi \alpha t' \qty[1 - e^{-R^2/4\alpha t'}] } \dd t' \nonumber \\
  =& \frac{\mu_aE_0}{2\rho c}e^{-\mu_a z} \int_0^t e^{\alpha t'\mu_a^2}  \nonumber \\
  \times & \qty[\erf\qty(\frac{z_0 + d - z}{\sqrt{4\alpha t'}} + \sqrt{\alpha t'}\mu_a ) - \erf\qty(\frac{z_0 - z}{\sqrt{4\alpha t'}} + \sqrt{\alpha t'}\mu_a ) ] \nonumber \\
  \times & \qty[1 - e^{-R^2/4\alpha t'} ]
  \dd t'
\end{align}
For a Gaussian beam we have
\begin{align}
  \label{eq:gaussian_temp}
  T(r =0,z,t) =& \frac{\mu_a}{\rho c} \int_0^t \qty(\frac{1}{4\pi \alpha t'})^{3/2} \qty{\int_{z_0}^{z_0 + d}e^{-\frac{(z-z')^2}{4\alpha t'}} e^{-\mu_a z'} \dd z'} \qty{2\pi\int_{0}^{R} E_0 e^{-\frac{r'^2}{\sigma^2}} e^{-\frac{r'^2}{4\alpha t'}}r'\dd r'}\dd t' \nonumber \\
  =& \frac{\mu_aE_0}{2\rho c}e^{-\mu_a z} \int_0^t e^{\alpha t'\mu_a^2}  \nonumber \\
  \times & \qty[\erf\qty(\frac{z_0 + d - z}{\sqrt{4\alpha t'}} + \sqrt{\alpha t'}\mu_a ) - \erf\qty(\frac{z_0 - z}{\sqrt{4\alpha t'}} + \sqrt{\alpha t'}\mu_a ) ] \nonumber \\
  \times & \frac{1}{1 + 4\alpha t'/\sigma^2} \qty[1 - e^{-R^2/4\alpha t'} ]
  \dd t'
\end{align}


\subsection{Approximations}

In theory, Equation \ref{eq:flattop_temp} and \ref{eq:gaussian_temp} can be integrated
numerically to calculate the temperature at $r = 0$ at any time $t$. However,
in practice the calculation is difficult. One of the issues is that the
terms arising from the $z$ integral become very large or small individually and
exceed the precision of standard floating point numbers for long time. This
is especially an issue when the absorption coefficient is large.

The $z$ portion of the integral is
\begin{equation*}
    \int_{z_0}^{z_0 + d} e^{-\frac{(z-z')^2}{4\alpha t'}} e^{-\mu_a z'} \dd z'
\end{equation*}
The Gaussian term can be written as a power series
\begin{equation*}
  e^{-\frac{(z-z')^2}{4\alpha t'}} = 1 - \frac{(z-z')^2}{4\alpha t'} + \frac{1}{2}\left(\frac{(z-z')^2}{4\alpha t'}\right)^2 + \ldots
\end{equation*}
If $(z-z')^2/4\alpha t'$ is small, which will be the case for positions $z$ close
to the source at long times, then
\begin{equation*}
    \int_{z_0}^{z_0 + d} e^{-\frac{(z-z')^2}{4\alpha t'}} e^{-\mu_a z'} \dd z' \approx
    \int_{z_0}^{z_0 + d} \left[1 - \frac{(z-z')^2}{4\alpha t'}\right] e^{-\mu_a z'} \dd z'
\end{equation*}
\begin{equation*}
    \left[1 - \frac{(z-z')^2}{4\alpha t'}\right] e^{-\mu_a z'}
=   \left[1 - \frac{z^2}{4\alpha t'} + \frac{2zz'}{4\alpha t'} -\frac{z'^2}{4\alpha t'}\right] e^{-\mu_a z'}
\end{equation*}
There are three integrals to evaluate
\begin{equation*}
    \int_{z_0}^{z_0 + d}
   \left(1 -  \frac{z^2}{4\alpha t'}\right)e^{-\mu_a z'}
    \dd z' =
    \left.\left(1 -  \frac{z^2}{4\alpha t'}\right)\frac{e^{-\mu_a z'}}{-\mu_a} \right|_{z_0}^{z_0 + d}
    = \left(1 -  \frac{z^2}{4\alpha t'}\right)\frac{e^{-\mu_a z_0} - e^{-\mu_a (z_0+d)}}{\mu_a}
\end{equation*}

\begin{equation*}
    \int_{z_0}^{z_0 + d}
   \frac{2zz'}{4\alpha t'} e^{-\mu_a z'}
   \dd z'
   = \left.\frac{2z}{4\alpha t'} \left(\frac{-\mu_az' - 1}{\mu_a^2} \right)e^{-\mu_a z'} \right|_{z_0}^{z_0 + d}
     = \frac{2z}{4\alpha t'} \left( \frac{\mu_a z_0 + 1}{\mu_a^2}e^{-\mu_a z_0} - \frac{\mu_a (z_0+d) + 1}{\mu_a^2}e^{-\mu_a (z_0+d)}\right)
\end{equation*}
\begin{align*}
    \int_{z_0}^{z_0 + d}
   \frac{z'^2}{4\alpha t'} e^{-\mu_a z'}
   \dd z'
   &= \left.\frac{1}{4\alpha t'} \left( \frac{z'^2}{-\mu_a} - \frac{2z'}{\mu_a^2} + \frac{2}{-\mu_a^3} \right)e^{-\mu_a z'} \right|_{z_0}^{z_0 + d} \\
   &= \left.\frac{1}{4\alpha t'} \left( \frac{z'^2}{\mu_a} + \frac{2z'}{\mu_a^2} + \frac{2}{\mu_a^3} \right)e^{-\mu_a z'} \right|_{z_0 + d}^{z_0} \\
   &=
   \frac{1}{4\alpha t'}\left[
   \left( \frac{z_0^2}{\mu_a} + \frac{2z_0}{\mu_a^2} + \frac{2}{\mu_a^3} \right)e^{-\mu_a z_0} -
   \left( \frac{(z_0 + d)^2}{\mu_a} + \frac{2(z_0 + d)}{\mu_a^2} + \frac{2}{\mu_a^3} \right)e^{-\mu_a (z_0+d)}
   \right]
\end{align*}


\begin{align*}
    \int_{z_0}^{z_0 + d} e^{-\frac{(z-z')^2}{4\alpha t'}} e^{-\mu_a z'} \dd z' &\approx
    \left(1 -  \frac{z^2}{4\alpha t'}\right)\frac{e^{-\mu_a z_0} - e^{-\mu_a (z_0+d)}}{\mu_a} \\
    &+ \frac{2z}{4\alpha t'} \left( \frac{\mu_a z_0 + 1}{\mu_a^2}e^{-\mu_a z_0} - \frac{\mu_a (z_0+d) + 1}{\mu_a^2}e^{-\mu_a (z_0+d)}\right) \\
    &- \frac{1}{4\alpha t'}\left[
   \left( \frac{z_0^2}{\mu_a} + \frac{2z_0}{\mu_a^2} + \frac{2}{\mu_a^3} \right)e^{-\mu_a z_0} -
   \left( \frac{(z_0 + d)^2}{\mu_a} + \frac{2(z_0 + d)}{\mu_a^2} + \frac{2}{\mu_a^3} \right)e^{-\mu_a (z_0+d)}
   \right]
\end{align*}
For $z=z_0=0$, this simplifies to
\begin{align*}
    \int_{z_0}^{z_0 + d} e^{-\frac{(z-z')^2}{4\alpha t'}} e^{-\mu_a z'} \dd z' &\approx
    \frac{1 - e^{-\mu_a d}}{\mu_a}
    - \frac{1}{4\alpha t'}\left[
   \left( \frac{2}{\mu_a^3} \right) -
   \left( \frac{d^2}{\mu_a} + \frac{2d}{\mu_a^2} + \frac{2}{\mu_a^3} \right)e^{-\mu_a d}
   \right]
\end{align*}
For ``long'' time, $4\alpha t' >> z$, we can keep just the first term,
\begin{align*}
    \int_{z_0}^{z_0 + d} e^{-\frac{(z-z')^2}{4\alpha t'}} e^{-\mu_a z'} \dd z' &\approx
    \frac{e^{-\mu_a z_0} - e^{-\mu_a (z_0+d)}}{\mu_a}
\end{align*}
Inserting the approximation into Equation \ref{eq:flattop_temp}
\begin{align}
  \label{eq:flattop_temp_approx}
  T(r=0, z,t) =& \frac{\mu_a E_0}{\rho c} \int_0^t \qty(\frac{1}{4\pi \alpha t'})^{3/2}  \nonumber \\
  \times & \left\{\left(1 -  \frac{z^2}{4\alpha t'}\right)\frac{e^{-\mu_a z_0} - e^{-\mu_a (z_0+d)}}{\mu_a} \right.\nonumber \\
    &+ \frac{2z}{4\alpha t'} \left( \frac{\mu_a z_0 + 1}{\mu_a^2}e^{-\mu_a z_0} - \frac{\mu_a (z_0+d) + 1}{\mu_a^2}e^{-\mu_a (z_0+d)}\right) \nonumber \\
    &- \left.\frac{1}{4\alpha t'}\left[
   \left( \frac{z_0^2}{\mu_a} + \frac{2z_0}{\mu_a^2} + \frac{2}{\mu_a^3} \right)e^{-\mu_a z_0} -
   \left( \frac{(z_0 + d)^2}{\mu_a} + \frac{2(z_0 + d)}{\mu_a^2} + \frac{2}{\mu_a^3} \right)e^{-\mu_a (z_0+d)}
 \right]
\right\} \nonumber \\
  \times & \qty{4\pi \alpha t' \qty[1 - e^{-R^2/4\alpha t'}] }
  \dd t' \\
  =&
  \frac{E_0}{\rho c} \int_0^t \qty(\frac{1}{4\pi \alpha t'})^{1/2}  \nonumber \nonumber \\
  \times & \left\{\qty[e^{-\mu_a z_0} - e^{-\mu_a (z_0+d)}]   -  \frac{z^2}{4\alpha t'}\qty[e^{-\mu_a z_0} - e^{-\mu_a (z_0+d)}]\right. \nonumber \\
    &+ \frac{2z}{4\alpha t'} \left( \frac{\mu_a z_0 + 1}{\mu_a}e^{-\mu_a z_0} - \frac{\mu_a (z_0+d) + 1}{\mu_a}e^{-\mu_a (z_0+d)}\right) \nonumber \\
    &- \left.\frac{1}{4\alpha t'}\left[
   \left( z_0^2       + \frac{2z_0}{\mu_a} + \frac{2}{\mu_a^2} \right)e^{-\mu_a z_0} -
   \left( (z_0 + d)^2 + \frac{2(z_0 + d)}{\mu_a} + \frac{2}{\mu_a^2} \right)e^{-\mu_a (z_0+d)}
 \right]
\right\} \nonumber \\
  \times & \qty[1 - e^{-R^2/4\alpha t'}]
  \dd t'
\end{align}

\subsection{Off Axis Temperatures}
The derivation of Equations \ref{eq:flattop_temp} and \ref{eq:gaussian_temp} assumed $x = y = 0$, i.e., it is only valid for the temperature rise on the $z$ axis.
To compute the temperature off axis, we need to evaluate
\begin{equation*}
  \int_{-\infty}^\infty \int_{-\infty}^\infty \bar{E}(x',y') e^{-\frac{(x-x')^2 + (y-y')^2}{4\alpha t'}} \dd x' \dd y'.
\end{equation*}
We can switch to polar coordinates by noting that $(x-x')^2 + (y-y')^2 = |\vec{r}-\vec{r}'|^2 = (\vec{r}-\vec{r}')\cdot(\vec{r}-\vec{r}') = \vec{r}\cdot\vec{r} + \vec{r}'\cdot\vec{r}' - 2r r' \cos(\phi - \phi')$
\begin{equation*}
  \int_0^\infty \int_0^{2\pi} \bar{E}(r',\phi')  e^{-\frac{r^2}{4\alpha t'}}e^{-\frac{r'^2}{4\alpha t'}}e^{\frac{2rr'\cos(\phi-\phi')}{4\alpha t'}}r' \dd \phi' \dd r'.
\end{equation*}
If the source term is symmetric about the $z$ axis, we can evaluate the integral at $\phi=0$ without loss of generality.
\begin{equation*}
\int_0^\infty \bar{E}(r') r'e^{-\frac{r^2}{4\alpha t'}}e^{-\frac{r'^2}{4\alpha t'}}\int_0^{2\pi} e^{2rr'\cos(\phi')/4\alpha t'} \dd \phi' \dd r'.
\end{equation*}
That the zero'th order Modified Bessel Function of the First Kind, $I_0(x)$ has an integral representation
\begin{equation}
  I_0(x) = \frac{1}{\pi} \int_0^\pi e^{x\cos(\phi)} \dd \phi.
\end{equation}
The azimuthal integral then can be carried out by noting that the integrand is symmetric about $\phi' = \pi$
\begin{equation*}
\int_0^{2\pi}  e^{2rr'\cos(\phi')/4\alpha t'}\dd \phi'
=2\int_0^{\pi}  e^{2rr'\cos(\phi')/4\alpha t'}\dd \phi'
=2\pi I_0\qty(\frac{2rr'}{4\alpha t'}).
\end{equation*}
The radian integral is then
\begin{equation*}
  2\pi \int_0^\infty \bar{E}(r')   e^{-\frac{r^2}{4\alpha t'}}e^{-\frac{r'^2}{4\alpha t'}}I_0\qty(\frac{2rr'}{4\alpha t'}) r'\dd r'.
\end{equation*}

\subsubsection{Flat Top Beams}
For a flat top beam with radius $R$, we need to evaluate the integral
\begin{equation*}
  2\pi \int_0^R  e^{-\frac{r^2}{4\alpha t'}}e^{-\frac{r'^2}{4\alpha t'}}I_0\qty(\frac{2rr'}{4\alpha t'}) r'\dd r'.
\end{equation*}
The integral can be carried out using the the Marcum Q-function, which is defined as
\begin{equation*}
Q_\nu(a,b) = 1 - \frac{1}{a^{\nu-1}}\int_0^b x^\nu e^{-\frac{x^2 + a^2}{2}} I_{\nu-1}(ax) \dd x.
\end{equation*}
In our case, $\nu=1$.
To cast our integral into this form, let $x = r'/\sqrt{2\alpha t'}, a = r/\sqrt{2\alpha t'}$, which gives
\begin{equation*}
  2\pi 2\alpha t' \int_0^{R/\sqrt{2\alpha t'}} e^{-a^2/2}e^{-x^2/2}I_0\qty(ax) x\dd x= 4\pi\alpha t' \qty(1 - Q_1(r/\sqrt{2\alpha t'},R/\sqrt{2\alpha t'}))
\end{equation*}
Plugging this into Equation \ref{eq:flattop_temp} gives
\begin{align}
  \label{eq:flattop_off_axis_temp}
  T(r=0, z,t) =& \frac{\mu_aE_0}{2\rho c}e^{-\mu_a z} \int_0^t e^{\alpha t'\mu_a^2}  \nonumber \\
  \times & \qty[\erf\qty(\frac{z_0 + d - z}{\sqrt{4\alpha t'}} + \sqrt{\alpha t'}\mu_a ) - \erf\qty(\frac{z_0 - z}{\sqrt{4\alpha t'}} + \sqrt{\alpha t'}\mu_a ) ] \nonumber \\
  \times & \qty[1 - Q_1\qty(r/\sqrt{2\alpha t'},R/\sqrt{2\alpha t'}) ]
  \dd t'
\end{align}



\subsubsection{Gaussian Beams}
For a Gaussian beam with 1/e radius $\sigma$ we need to evaluate
\begin{equation*}
  2\pi \int_0^\infty
  e^{-\frac{r'^2}{\sigma^2}}
  e^{-\frac{r^2}{4\alpha t'}}e^{-\frac{r'^2}{4\alpha t'}}I_0\qty(\frac{2rr'}{4\alpha t'}) r'\dd r'
  =
  2\pi e^{-\frac{r^2}{4\alpha t'}}\int_0^\infty
  e^{-\qty(\frac{1}{\sigma^2} + \frac{1}{4\alpha t'})r'^2}
  I_0\qty(\frac{2rr'}{4\alpha t'}) r'\dd r'.
\end{equation*}
% The integral can be cast into a standard form and evaluated,
% \begin{equation*}
%   \int_0^\infty t^{\nu + 1} I_\nu\qty(bt) e^{-p^2 t^2} \dd t  = \frac{b^\nu}{\qty(2p^2)^{\nu+1}}  e^{b^2/4p^2}.
% \end{equation*}
% In our case, $\nu = 0$, $b = 2r/4\alpha t'$, and $p^2 = \qty(\frac{1}{\sigma^2} + \frac{r'^2}{4\alpha t'})$, which gives
% \begin{align*}
%   2\pi e^{-\frac{r^2}{4\alpha t'}}\int_0^\infty
%   e^{-\qty(\frac{1}{\sigma^2} + \frac{1}{4\alpha t'})r'^2}
%   I_0\qty(\frac{2rr'}{4\alpha t'}) r'\dd r'
%   &=
%   2\pi e^{-\frac{r^2}{4\alpha t'}}
%   \frac{1}{2\qty(\frac{1}{\sigma^2} + \frac{1}{4\alpha t'})}  e^{\qty(2r/4\alpha t')^2/4\qty(\frac{1}{\sigma^2} + \frac{1}{4\alpha t'})} \\
%   &=
%   4\pi\alpha t' e^{-\frac{r^2}{4\alpha t'}}
%   \frac{1}{1 + \frac{4\alpha t'}{\sigma^2}}
%   e^{\frac{r^2}{4\alpha t'} \frac{1}{ 1 + \frac{4\alpha t'}{\sigma^2}} }
% \end{align*}
The integral can be cast into a standard form and evaluated [``Table of Integrals, Series, and Products'', Gradshteyn and Ryzhik pg. 707],
\begin{equation*}
  \int_0^\infty x e^{-\alpha x^2} I_\nu\qty(\beta x) J_\nu(\gamma x) \dd x = \frac{1}{2\alpha} \exp\qty(\frac{\beta^2 - \gamma^2}{4\alpha}) J_\nu\qty(\frac{\beta\gamma}{2\alpha})
\end{equation*}
In our case, $\nu = \gamma = 0$, $\beta = 2r/4\alpha t'$, and $\alpha = \qty(\frac{1}{\sigma^2} + \frac{1}{4\alpha t'})$, which gives
\begin{align*}
  2\pi e^{-\frac{r^2}{4\alpha t'}}\int_0^\infty
  e^{-\qty(\frac{1}{\sigma^2} + \frac{1}{4\alpha t'})r'^2}
  I_0\qty(\frac{2rr'}{4\alpha t'}) r'\dd r'
  &=
  2\pi e^{-\frac{r^2}{4\alpha t'}}
  \frac{1}{2\qty(\frac{1}{\sigma^2} + \frac{1}{4\alpha t'})}  e^{\qty(2r/4\alpha t')^2/4\qty(\frac{1}{\sigma^2} + \frac{1}{4\alpha t'})} \\
  &=
  4\pi\alpha t' e^{-\frac{r^2}{4\alpha t'}}
  \frac{1}{1 + \frac{4\alpha t'}{\sigma^2}}
  e^{\frac{r^2}{4\alpha t'} \frac{1}{ 1 + \frac{4\alpha t'}{\sigma^2}} } \\
  &=
  4\pi\alpha t'
  \frac{1}{1 + \frac{4\alpha t'}{\sigma^2}}
  e^{\frac{r^2}{4\alpha t'} \qty(\frac{1}{ 1 + \frac{4\alpha t'}{\sigma^2}}-1) } \\
  &=
  4\pi\alpha t'
  \frac{\sigma^2}{\sigma^2 + 4\alpha t'}
  e^{\frac{-r^2}{\sigma^2 + 4\alpha t'}  }
\end{align*}
Plugging this into Equation \ref{eq:gaussian_temp} gives
\begin{align}
  \label{eq:gaussian_off_axis_temp}
  T(r=0, z,t) =& \frac{\mu_aE_0}{2\rho c}e^{-\mu_a z} \int_0^t e^{\alpha t'\mu_a^2}  \nonumber \\
  \times & \qty[\erf\qty(\frac{z_0 + d - z}{\sqrt{4\alpha t'}} + \sqrt{\alpha t'}\mu_a ) - \erf\qty(\frac{z_0 - z}{\sqrt{4\alpha t'}} + \sqrt{\alpha t'}\mu_a ) ] \nonumber \\
  \times & \qty[
  \frac{\sigma^2}{\sigma^2 + 4\alpha t'}
  e^{\frac{-r^2}{\sigma^2 + 4\alpha t'}  }
  ]
  \dd t'
\end{align}



\subsection{Pulsed Exposures}

The temperature rises caused by a CW exposure is given by integrating the Green's Function,
\begin{equation}
  \Delta T(t) = \int_{0}^{t} G(t') \dd t'.
\end{equation}
Because the temperature rise is linear, we can use the CW temperature rise to compute
the temperature rise caused by a pulsed exposure,
\begin{equation}
  \Delta T(t) = \begin{cases}
    \int_{0}^{t} G(t') \dd t' & t \le \tau \\
    \int_{0}^{t} G(t') \dd t' - \int_{0}^{t-\tau} G(t') \dd t' & t > \tau.
\end{cases}
\end{equation}
After the pulse, the temperature rise is given by the difference between the CW temperature rise at the times $t$ and $t-\tau$.
We can either calculate $\Delta T(t)$ and $\Delta(t-\tau)$, or evaluate the integral over a different set of limits,
\begin{equation}
    \int_{0}^{t} G(t') \dd t' - \int_{0}^{t-\tau} G(t') \dd t'  = \Delta T(t) - \Delta T(t-\tau) = \int_{t-\tau}^{t} G(t') \dd t'
\end{equation}





\end{document}
